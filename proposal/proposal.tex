\documentclass[11pt,twoside]{article}
\usepackage[authoryear,semicolon]{natbib}
\usepackage[T1]{fontenc}
\usepackage[utf8]{inputenc}
\usepackage{textgreek}
\usepackage[color=yellow]{todonotes}
\setlength{\marginparwidth}{1.25in}
\usepackage{xifthen}
\usepackage{fullpage}
\usepackage{amsmath}
\usepackage{amssymb}
\usepackage{stmaryrd}
\usepackage{amsthm}
\usepackage{mathtools}
\usepackage{listings}
\usepackage{bussproofs}
\usepackage{proof}
\usepackage{colonequals}
\usepackage{comment}
\usepackage{textcomp}
\usepackage[us]{optional}
\usepackage{color}
\usepackage{url}
\usepackage{verbatim}
\usepackage{graphics}
\usepackage{mathpartir}
\usepackage{import}
\usepackage{stackengine}
\usepackage{scalerel}
\usepackage{xcolor}

\begin{document}
\title{CIS 670 Project Proposal}
\author{Paul He, Irene Yoon}
\date{Spring 2021}

\maketitle{}


\section{Ideas / Brainstorming}
Curiosities related to Iris that may or may not lead to a 
project.
\begin{enumerate} 
    \item Pi-calculus encoding in Iris \\
    - Possible direction: Use session-typed $\pi$-calculus encoding in Actris
    and see if it's compatible with the syntactic translation of game 
    semantics proposed by Yoshida et al. It might be interesting to see
    if we can translate properties that are "game semantic" onto this 
    framework.
    For instance, we could imagine stating invariants that are on the game 
    semantic language proposed in Yoshida et al. and providing a syntactic 
    translation into the $pi$-calculus terms. We would be able to write 
    "high-level" specifications in the game semantic language, and then
    the proofs will be in the "low-level" $\pi$-calculus framework, which
    are easier to reason about?
    \item CBPV $\lambda$-calculus in Iris, leveraging logical relations to
    define contextual equivalence in this language. This work is already 
    formalized in Coq by C. Rizkallah et al., and would be nice because
    we can ask Steve on technical details if we're confused about anything.
    \item Complexity analysis of concurrent algorithms using Iris

    ==== More open-ended questions ====
    \item Quantum programming in Iris?
    \item Functional/proof pearl ideas in Iris?
    \item OK, higher-order stores are neat, but are there other higher-order
    data structures that we might care about? Does Iris give us any
    advantages here? 
\end{enumerate}




\end{document}
